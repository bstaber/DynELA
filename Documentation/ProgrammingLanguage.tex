% !TeX spellcheck = en_US
% !TeX root = DynELA.tex
%
% LaTeX source file of DynELA FEM Code
%
% (c) by Olivier Pantalé 2020
%
\chapter{DynELA programming language}

\startcontents[chapters]
\printmyminitoc[1]\LETTRINE{T}his chapter deals about the \DynELA~programming language. This language is based on Python 3 and all models must be described using this formalism. Therefore, this chapter will describe step by step how to build a Finite Element Model for the \DynELA, using the Python 3 language.

\section{Introduction and basic knowledge}

\subsection{Calling the python interpreter}

After the installation and compilation phase of the code\footnote{See the installation instructions in chapter \ref{Chapter!Installation} of the preamble, page \pageref{Chapter!Installation}}, the \DynELA~can be run using the following command:

\begin{BashListing}
python model.py
\end{BashListing}

where \textsf{model.py} is the Python 3 source file defining the Finite Element Model. The \textsf{model.py} file contains the definition of the Finite Element Model using a Python 3 language and calls to specific DynELA methods written in \Cpp.

\subsection{Formalism of a DynELA python file}

To build a Finite Element Model, it is mandatory to import the \textsf{dnlPython} interpreter from the \textsf{.py} script. Conforming to this formalism, we give hereafter the minimal piece of Python code to set up a Finite Element Model in the \DynELA.

\begin{PythonListing}
#!/usr/bin/env python3
import dnlPython as dnl # Imports the dnlPython library as dnl
model = dnl.DynELA()    # Creates the main Object
...                     # Set of instructions to build the FE model
...                     # conforming to the DynELA language and Python 3
model.solve()           # Runs the solver
...                     # Set of instructions to postprocess the FE model
\end{PythonListing}

In the preceding piece of code, line 2 is used to load into the namespace \textsf{dnl} the \textsf{dnlPython} module containing the interface to all \Cpp~methods of the \DynELA, based on the use of the SWIG Python interface. Therefore, all public methods of the \DynELA~written in \Cpp~can be called from the Python script to build the Finite Element Model, launch the solver, produce output results,\ldots

In the proposed piece of code, line 3 is used to create an object of type \textsf{DynELA} (the higher object type in the \DynELA~library) and instantiate it as the \textsf{model} object\footnote{For the rest of this chapter, we assume that the name of the instantiated \textsf{DynELA} object is \textsf{model.}}, while line 6, the solver of the \DynELA~library is called to solve the problem and produce the results.

As the interpreter of the \DynELA~is based on Python 3 language, we can use all instructions valid in Python 3 along with the specific DynELA instructions. In the rest of this documentation, we assume that the notions of programming in Python are mastered, and we will focus only on the functions specific to the \DynELA.

\section{The Kernel library}

\#include "LogFile.h"

\#include "MacAddress.h"

\#include "Settings.h"

\#include "String.h"

\#include "System.h"

\#include "Timer.h"

\#include "Field.h"

\section{The Maths library}

\#include "DiscreteFunction.h"

\#include "DiscreteFunctionSet.h"

\#include "Function.h"

\#include "Matrices.h"

\#include "Matrix.h"

\#include "MatrixDiag.h"

\#include "PolynomialFunction.h"

\#include "RampFunction.h"

\#include "SinusFunction.h"

\#include "SymTensor2.h"

\#include "Tensor2.h"

\#include "Tensor3.h"

\#include "Tensor4.h"

\#include "Vec3D.h"

\#include "Vector.h"

\#include "ColorMap.h"

\section{Model, Nodes and Elements}

All Finite Element Models involves nodes and elements. The very first part of the model is therefore to create the nodes and the elements of the structure to set up a Finite Element Model. The \DynELA~library doesn't include any meshing procedure yet, therefore, it is mandatory to create all elements and all nodes by hand or using Python loops in case it can be used. Another way is to use an external meshing program and convert the output of this program to produce the ad hoc lines of Python to describe the elements and the nodes of the model. This has been used many times by the author, and the Abaqus Finite Element code is an efficient way to create the mesh using the \textsf{.inp} text file generated by the CAE Abaqus program.

\subsection{Model}

Definition of a model in the \DynELA~is done by creating an instance of the \textsf{DynELA} object into memory. This is done by calling the \textsf{dnlPython.DynELA()} method that returns an object of type \textsf{DynELA} as presented hereafter.

\begin{PythonListing}
import dnlPython as dnl       # Imports the dnlPython library as dnl
model = dnl.DynELA("Taylor")  # Creates the main Object model named Taylor
\end{PythonListing}

In line 2 of the preceding piece of code, a reference name\footnote{The reference name is a string used to identify the object, this is completely optional but useful for debugging purposes for example as one can know the associated name to an object.} \textsf{Taylor} is associated to the model object during creation. Once the model is created, one can then define all nodes, elements, materials, constitutive laws, boundary conditions,\ldots

\subsection{Nodes}

\subsubsection{Definition of the nodes}

In the \DynELA, creation of nodes is done by calling the \textsf{DynELA.createNode()} method. Therefore, a node is created by calling the \textsf{createNode()} method and giving the new node number and the $x$, $y$ and $z$ coordinates of the new node as presented just below.

\begin{PythonListing}
model.createNode(1, 0.0, 0.0, 0.0)  # Creates node 1, coordinates [0.0, 0.0, 0.0]
model.createNode(2, 1.0, 2.0, -1.0) # Creates node 2, coordinates [1.0, 2.0, -1.0]
\end{PythonListing}

An alternative method can be used if the coordinates of the node are already stored into a Vec3D object as presented hereafter.

\begin{PythonListing}
vect = dnl.Vec3D(1.0, 2.0, -1.0) # Creates a Vec3D object [1.0, 2.0, -1.0]
model.createNode(1, vect)        # Creates node 1 with coordinates vect
\end{PythonListing}

A check of the total number of nodes of the structure can be done using the \textsf{DynELA.getNodesNumber()} method that returns the total number of nodes created.

\subsubsection{Definition of the Nodes sets}

Manipulation of nodes, application of boundaries conditions, etc,\ldots is done through the definition of nodes sets. Such nodes sets are used to group nodes under a \textsf{NodeSet} object for further use. A \textsf{NodeSet} object contains a reference name and a list of nodes. Creation of a \textsf{NodeSet} is done using the \textsf{DynELA.NodeSet()} method that returns an new \textsf{NodeSet} instance. The \textsf{NodeSet} can be named during the creation by specifying its name as a string.

\begin{PythonListing}
nset = dnl.NodeSet("NS_All")
\end{PythonListing}

When the \textsf{NodeSet} has been created, one can now define the list of nodes constituting the \textsf{NodeSet} with the generic \textsf{DynELA.add()} method with the following formalism:

\textsf{DynELA.add(nodeset, start, end, increment)}

Hereafter is some self explaining examples to illustrate this process.

\begin{PythonListing}
nset = dnl.NodeSet("NS_All")
model.add(nset, 2)       # Add node number 2 to node set
model.add(nset, 1, 4)    # Add nodes number 1-4 to node set
model.add(nset, 1, 4, 2) # Add nodes number 1 and 3 to node set
\end{PythonListing}

\subsection{Elements}

\subsubsection{Definition of the elements}

Creation of elements is done by calling the \textsf{DynELA.createElement()} method. An element is created by calling the \textsf{createNode()} method and giving the new element number and the list of nodes defining the element shape separated by comas and ordered thanks to the element definition as presented just hereafter.

\textsf{DynELA.createElement(elementNumber, node1, node2,\ldots)}

Before creating the very first element of the structure, it is necessary to define the element shape using the \textsf{DynELA.setDefaultElement()} method. An example of element creation combining the two preceding methods is presented hereafter.

\begin{PythonListing}
model.setDefaultElement(dnl.Element.ElQua4N2D) # Defines the default element
model.createElement(1, 1, 2, 3, 4)             # Creates element 1 with nodes 1,2,3,4
\end{PythonListing}

The following elements are available in the \DynELA.
\begin{description}
\item [ElQua4n2D]: 4 nodes bi-linear 2D quadrilateral element.
\item [ElQua4NAx]: 4 nodes bi-linear axisymmetric quadrilateral element.
\item [ElTri3N2D]: 3 nodes 2D triangular element.
\item [ElHex8N3D]: 8 nodes 3D hexahedral element.
\item [ElTet4N3D]: 4 nodes 3D tetrahedral element.
\item [ElTet10N3D]: 10 nodes 3D tetrahedral element.
\end{description}
The total number of elements of the structure can be checked using the \textsf{DynELA.getElementsNumber()} method that returns the total number of elements created.

\subsubsection{Definition of the Element sets}

Declaration of materials, boundaries conditions, etc\ldots is done through the definition of elements sets. Such elements sets are used to group elements under an \textsf{ElementSet} object for further use. An \textsf{ElementSet} object contains a reference to a name and a list of elements. Creation of an \textsf{ElementSet} is done using the \textsf{DynELA.ElementSet()} method that returns a new \textsf{ElementSet} instance. The \textsf{ElementSet} can be named during the creation by specifying its name as a string.

\begin{PythonListing}
eset = dnl.ElementSet("ES_All")
\end{PythonListing}

When the \textsf{ElementSet} has been created, one can now define the list of elements constituting the \textsf{ElementSet} with the generic \textsf{DynELA.add()} method according to the following formalism:

\textsf{DynELA.add(elementset, start, end, increment)}

Hereafter is some self explaining examples to illustrate this process.

\begin{PythonListing}
eset = dnl.ElementSet("ES_All")
model.add(eset, 2)       # Add element number 2 to element set
model.add(eset, 1, 4)    # Add elements number 1-4 to element set
model.add(eset, 1, 4, 2) # Add elements number 1 and 3 to element set
\end{PythonListing}

\subsection{Coordinates transformations}

When the mesh has been created, it is always possible to modify the geometry of the structure by applying some geometrical operations such as translations, rotations and change of scale. Those operations apply on a \textsf{NodeSet}.

\subsubsection{Translations}

One can define a translation of the whole model or a part of the model by defining a translation vector (an instance of the \DynELA~\textsf{Vec3D}) and apply this translation to the whole structure (without specifying the \textsf{NodeSet}) or a \textsf{NodeSet} using the \textsf{DynELA.translate()} method with the following syntax.

\begin{PythonListing}
vector = dnl.Vec3D(1, 0, 0)   # Defines the translation vector
model.translate(vector)       # Translates the whole model along [1, 0, 0]
model.translate(vector, nset) # Translates the NodeSet nset along [1, 0, 0]
\end{PythonListing}

\subsubsection{Rotations}

One can define a rotation of the whole model or a part of the model by defining a rotation vector (global axes $\overrightarrow{\ensuremath{x}}$, $\overrightarrow{y}$, $\overrightarrow{\ensuremath{z}}$ or an instance of the \DynELA~\textsf{Vec3D}) and an angle $\alpha$ then apply this rotation to the whole structure (without specifying the \textsf{NodeSet}) or a \textsf{NodeSet} using the \textsf{DynELA.rotate()} method with the following syntax.

\begin{PythonListing}
model.rotate('X', angle)        # Rotation of the whole structure around X
model.rotate('X', angle, nset)  # Rotation of NodeSet nset around X
axis = dnl.Vec3D(1.0, 1.0, 1.0) # Defines the axis of rotation
model.rotate(axis, angle)       # Rotation of the whole structure around axis
model.rotate(axis, angle, nset) # Rotation of NodeSet nset around axis
\end{PythonListing}

\subsubsection{Scaling}

One can define a scaling of the whole model or a part of the model by defining a scale factor or a scale vector (an instance of the \DynELA~\textsf{Vec3D}) and apply this scaling operation to the whole structure (without specifying the \textsf{NodeSet}) or a \textsf{NodeSet} using the \textsf{DynELA.scale()} method with the following syntax.

\begin{PythonListing}
model.scale(value)             # Scales the whole structure by factor value
model.scale(value, nset        # Scales the NodeSet nset by factor value
vec = dnl.Vec3D(2.0, 1.0, 1.0) # Defines the scale vector
model.scale(vec)               # Scales the whole structure by a factor of 2.0 on x
model.scale(vec, nset)         # Scales the NodeSet nset by a factor of 2.0 on x
\end{PythonListing}

\section{Materials}

\subsection{Declaration of materials}

\subsubsection{Material declaration}

Creation of a Material is done using the \textsf{DynELA.Material()} method. It is possible to give a name to a material during the creation process by specifying it through a string during the declaration. This can be used further.

\begin{PythonListing}
# Creates the material
steel = dnl.Material("Steel")
\end{PythonListing}

\subsubsection{General properties of materials}

General properties of materials in \DynELA~concerns the general constants such as Young's modulus, Poisson's ratio, density,\ldots The complete list of parameters is reported in Table \ref{tab:Programming!GeneralProperties}.
\begin{table}[h]
	\begin{center}\begin{tcolorbox}[width=.75\textwidth,myTab,tabularx={l|c|c|R}]
			\multicolumn{1}{c|}{Name} & Symbol & Unit & \multicolumn{1}{c}{Description} \\ \hline\hline
			youngModulus & $E$ & $MPa$ & Young modulus\\
			poissonRatio & $\nu$ &  & Poisson ratio\\
			density & $\rho$ & $kg/m^3$ & Density\\
			heatCapacity & $C_{p}$ & $J/^{\circ}C$ & Heat capacity\\
			taylorQuinney& $\eta$ & & Taylor-Quinney coefficient\\
			initialTemperature & $T_{0}$ & $^{\circ}C$ & Initial temperature
	\end{tcolorbox}\end{center}\caption{General properties of materials\label{tab:Programming!GeneralProperties}}
\end{table}
After creating an instance of the object \textsf{dnl.Material}, on can apply the prescribed values to all those parameters using the following syntax.

\begin{PythonListing}
# Creates the material
steel = dnl.Material("Steel")
# Apply all parameters
steel.youngModulus = 206e9
steel.poissonRatio = 0.3
steel.density = 7830
steel.heatCapacity = 46
steel.taylorQuinney = 0.9
steel.initialTemperature = 25
\end{PythonListing}

\subsubsection{Material affectation to a set of elements}

And, the material can be affected to the elements of the model by the \textsf{DynELA.add()} method as proposed hereafter.

\begin{PythonListing}
# Creates the material
steel = dnl.Material("Steel")
# Apply all parameters
...
# Affect the material to the element set eset
model.add(steel, eset)
\end{PythonListing}

\subsection{Johnson-Cook constitutive law}

The Johnson-Cook constitutive law is an hardening law defining the yield stress $\sigma^{y}(\overline{\varepsilon}^{p},\stackrel{\bullet}{\overline{\varepsilon}^{p}},T)$ by the following equation:

\begin{equation}
\sigma^{y}=\left(A+B\overline{\varepsilon}^{p^{n}}\right)\left[1+C\ln\left(\frac{\stackrel{\bullet}{\overline{\varepsilon}^{p}}}{\stackrel{\bullet}{\overline{\varepsilon}_{0}}}\right)\right]\left[1-\left(\frac{T-T_{0}}{T_{m}-T_{0}}\right)^{m}\right]
\end{equation}
where $\stackrel{\bullet}{\overline{\varepsilon}_{0}}$ is the reference strain rate, $T_{0}$ and $T_{m}$ are the reference temperature and the melting temperature of the material respectively and $A$, $B$, $C$, $n$ and $m$ are the five constitutive flow law parameters. Therefore, this kind of hardening law can be defined by using the following piece of code:

\begin{PythonListing}
hardLaw = dnl.JohnsonCookLaw()                       # Hardening law
hardLaw.setParameters(A, B, C, n, m, depsp0, Tm, T0) # Parameters of the law
\end{PythonListing}

Once the hardening law has been created, one have to link this hardening law to a material already defined using the following piece of code:

\begin{PythonListing}
# Creates the material
steel = dnl.Material("Steel")
# Creates the hardening law
hardLaw = dnl.JohnsonCookLaw()
# Attach hardening law to material
steel.setHardeningLaw(hardLaw)
\end{PythonListing}

\section{Boundaries conditions}

\subsection{Restrain boundary condition}

\begin{PythonListing}
# Declaration of a boundary condition for top part
topBC = dnl.BoundaryRestrain('BC_top')
topBC.setValue(0, 1, 1)
model.attachConstantBC(topBC, topNS)
\end{PythonListing}

\subsection{Amplitude}

\begin{PythonListing}
# Declaration of a ramp function to apply the load
ramp = dnl.RampFunction("constantFunction")
ramp.set(dnl.RampFunction.Constant, 0, stopTime)
\end{PythonListing}

\subsection{Constant speed}

\begin{PythonListing}
# Declaration of a boundary condition for top part
topSpeed = dnl.BoundarySpeed()
topSpeed.setValue(displacement, 0, 0)
topSpeed.setFunction(ramp)
model.attachConstantBC(topSpeed, topNS)
\end{PythonListing}

\subsection{Initial speed}

\begin{PythonListing}
# Declaration of a ramp function to apply the load
ramp = dnl.RampFunction("constantFunction")
ramp.set(dnl.RampFunction.Constant, 0, stopTime)
\end{PythonListing}

\section{Fields}

\subsection{Nodal fields}

Nodal fields are defined at nodes and cover types defined in table \ref{tab:Programming!NodalFields}. Concerning those fields, some of them are directely defined at nodes, some other are extrapolated from integration points and transfered to nodes as reported in column \textsf{loc} of table \ref{tab:Programming!NodalFields}. Concerning types, \textsf{scalars}, \textsf{vec3D} and \textsf{tensors} are available. Depending in the type of data, different methods can be used to acces those data:
\begin{description}
	\item [{scalar}] : Direct access to the value as it is unique.
	\item [{vec3D}] : Access to all $3$ components of a vec3D using \textsf{nameX}, \textsf{nameY}, \textsf{nameZ} or the norm of the vec3D using \textsf{name}.
	\item [{tensor}] : Access to all $9$ components of a tensor using \textsf{nameXX}, \textsf{nameXY},\ldots, \textsf{nameZZ} or the norm of the tensor using \textsf{name}.
\end{description}
\begin{table}[h]
	\begin{center}\begin{tcolorbox}[width=.75\textwidth,myTab,tabularx={l|c|c|c|R}]
			\multicolumn{1}{c|}{Name} & Type & Label & Loc & \multicolumn{1}{c}{Description} \\ \hline\hline
			density & scalar & & IntPt &\\ \hline
			displacementIncrement & vec3D & & node & \\ \hline
			displacement & vec3D & & node & \\ \hline
			energyIncrement & scalar & & IntPt & \\ \hline
			energy & scalar & & IntPt & \\ \hline
			gammaCumulate & scalar & & IntPt & \\ \hline
			gamma & scalar & & IntPt & \\ \hline
			internalEnergy & scalar & & IntPt & \\ \hline
			mass & scalar & & node & \\ \hline
			nodeCoordinate & vec3D & & node & \\ \hline
			normal & vec3D & & node & \\ \hline
			PlasticStrainInc & tensor & & IntPt & \\ \hline
			plasticStrainRate & scalar & & IntPt & \\ \hline
			plasticStrain & scalar & & IntPt & \\ \hline
			PlasticStrain & tensor & & IntPt & \\ \hline
			pressure & scalar & & IntPt & \\ \hline
			speedIncrement & vec3D & & node & \\ \hline
			speed & vec3D & & node & \\ \hline
			StrainInc & tensor & & IntPt & \\ \hline
			Strain & tensor & & IntPt & \\ \hline
			Stress & tensor & & IntPt & \\ \hline
			temperature & scalar & & IntPt & \\ \hline
			vonMises & scalar & & IntPt & \\ \hline
			yieldStress & scalar & & IntPt & 
	\end{tcolorbox}\end{center}\caption{Nodal fields\label{tab:Programming!NodalFields}}
\end{table}


\subsection{Element fields}

Element fields are defined at integration points and cover types defined in table \ref{tab:Programming!ElementlFields}. Concerning types, \textsf{scalars}, \textsf{vec3D} and \textsf{tensors} are available. Depending in the type of data, different methods can be used to acces those data:
\begin{description}
	\item [{scalar}] : Direct access to the value as it is unique.
	\item [{vec3D}] : Access to all $3$ components of a vec3D using \textsf{nameX}, \textsf{nameY}, \textsf{nameZ} or the norm of the vec3D using \textsf{name}.
	\item [{tensor}] : Access to all $9$ components of a tensor using \textsf{nameXX}, \textsf{nameXY},\ldots, \textsf{nameZZ} or the norm of the tensor using \textsf{name}.
\end{description}
\begin{table}[h]
	\begin{center}\begin{tcolorbox}[width=.75\textwidth,myTab,tabularx={l|c|c|R}]
			\multicolumn{1}{c|}{Name} & Type & Label & \multicolumn{1}{c}{Description} \\ \hline\hline
			density & scalar & & \\ \hline
			gammaCumulate & scalar  & & \\ \hline
			gamma & scalar & & \\ \hline
			internalEnergy & scalar & & \\ \hline
			plasticStrainRate & scalar & & \\ \hline
			plasticStrain & scalar & & \\ \hline
			PlasticStrain & tensor & & \\ \hline
			PlaticStrainInc & tensor & & \\ \hline
			pressure & scalar & & \\ \hline
			StrainInc & tensor & & \\ \hline
			Strain & tensor & & \\ \hline
			Stress & tensor & & \\ \hline
			temperature & scalar & & \\ \hline
			vonMises & scalar & & \\ \hline
			yieldStress & scalar & &
	\end{tcolorbox}\end{center}\caption{Element fields\label{tab:Programming!ElementlFields}}
\end{table}

\subsection{Global fields}

\begin{table}[h]
	\begin{center}\begin{tcolorbox}[width=.75\textwidth,myTab,tabularx={l|c|c|R}]
			\multicolumn{1}{c|}{Name} & Type & Label & \multicolumn{1}{c}{Description} \\ \hline\hline
			kineticEnergy & scalar & & \\ \hline
			realTimeStep & scalar & & \\ \hline
			timeStep & scalar & & 
	\end{tcolorbox}\end{center}\caption{Global fields\label{tab:Programming!GlobalFields}}
\end{table}

\section{Data Output during computation}

\subsection{VTK Data files}

\begin{PythonListing}
model.setSaveTimes(0, stopTime, stopTime/nbreSaves)
\end{PythonListing}

\subsection{History files}

\begin{PythonListing}
dtHist = dnl.HistoryFile("dtHistory")
dtHist.setFileName(dnl.String("dt.plot"))
dtHist.add(dnl.Field.timeStep)
dtHist.setSaveTime(stopTime/nbrePoints)
model.add(dtHist)
\end{PythonListing}

\section{Solvers}

\begin{PythonListing}
# Declaration of the explicit solver
solver = dnl.Explicit("Solver")
solver.setTimes(0, stopTime)
solver.setTimeStepSafetyFactor(1.0)
model.add(solver)
\end{PythonListing}

\subsection{Parallel solver}

\begin{PythonListing}
# Parallel solver with two cores
model.parallel.setCores(2)
\end{PythonListing}

\subsection{Solving procedure}

\begin{PythonListing}
# Run the main solver
model.solve()
\end{PythonListing}

\section{Vectorial SVG contourplots}

