%
% macrosHyperlink TeX configuration file
%
% (c) by Olivier Pantalé 2020
%
\usepackage{kurier} % Set the default font for the document
%\usepackage{bookman}
\usepackage{helvet}
\usepackage[T1]{fontenc}
\usepackage[utf8]{inputenc}
\usepackage[a4paper]{geometry}
\geometry{verbose,tmargin=3cm,bmargin=2cm,lmargin=2cm,rmargin=1.5cm,headheight=1cm,headsep=1cm,footskip=1cm}
\pagestyle{headings}
\setcounter{secnumdepth}{4}
\setcounter{tocdepth}{4}
\setlength{\parskip}{\medskipamount}
\setlength{\parindent}{0pt}
\usepackage{color}
\usepackage{array}
\usepackage{babel}
\usepackage{longtable}
\usepackage{multirow}
\usepackage{float}
\usepackage{textcomp}
\usepackage{amsmath}
\usepackage{amssymb}
\usepackage{cancel}
\usepackage{stmaryrd}
\usepackage{graphicx}
\PassOptionsToPackage{normalem}{ulem}
\usepackage{ulem}
\usepackage{float}
\usepackage{cite}
\usepackage{geometry}
\usepackage{lettrine} 
\usepackage{pict2e}
\usepackage{minitoc}
\usepackage{comment}
\usepackage{tikz}
\usepackage{titletoc}
\usepackage{calc}
\usepackage{makeidx}
\usepackage[]{titlesec} 
\usepackage{placeins}
\usepackage{tcolorbox}
\usepackage{tabularx}
\usepackage{array}
\usepackage{colortbl}
\usepackage{fancyhdr}
\usepackage{tablefootnote}
\usepackage[nice]{nicefrac}
\usetikzlibrary{shadows.blur}
\tcbuselibrary{skins}

% Define the macros for color tables
\newcolumntype{L}{>{\raggedleft\arraybackslash}X}
\newcolumntype{R}{>{\raggedright\arraybackslash}X}
\newcolumntype{C}{>{\centering\arraybackslash}X}
\tcbset{myTab/.style={enhanced,fonttitle=\bfseries,fontupper=\normalsize,
colback=gray!10!white,colframe=gray!85!black,colbacktitle=gray,
coltitle=white,center title}}

% Sets the caption of all figures
\usepackage{caption}
\captionsetup[figure]{labelfont={bf,it}, textfont=it}
\captionsetup[table]{labelfont={bf,it}, textfont=it}

% Change the list spaces
\usepackage{enumitem}
\setlist[description]{itemsep=0.3em, parsep=0em, itemindent=-1em}

\makeatletter

% Titre de l'ouvrage
\newcommand{\MyTitle}{Academic year 2020/2021}

%Command for bilingual texts
\newcommand{\frEn}[2]{#2} % Default English

% pour les doubles versions des documents
\newenvironment{fullVersion}{}{}

% Dimensions de la page
\geometry{verbose,a4paper,tmargin=3cm,bmargin=2.5cm,lmargin=2cm,rmargin=2cm,headheight=1cm,headsep=1cm,footskip=1cm}

\pagestyle{fancyplain}
\renewcommand{\floatpagefraction}{1.0}
\renewcommand{\textfraction}{0.2}
\renewcommand{\bottomfraction}{1.0}
\setcounter{tocdepth}{2}
\setcounter{secnumdepth}{3}

% Lettrine
\newcommand{\LETTRINE}[1]{\lettrine[lines=3]{\textcolor{myRed}{#1}}{}}

% style spcial pour les footnotes
\renewcommand\@makefnmark{\@textsuperscript{\normalfont(\@thefnmark)}}

% style special pour les rfrences personnelles
\newcommand{\myCite}[1]{\textbf{\underbar{\cite{#1}~}}}

% modification de la bibliographie
\renewenvironment{thebibliography}[1]
     {\section*{\bibname}%
      \@mkboth{\MakeUppercase\bibname}{\MakeUppercase\bibname}%
      \list{\@biblabel{\@arabic\c@enumiv}}%
           {\settowidth\labelwidth{\@biblabel{#1}}%
            \leftmargin\labelwidth
            \advance\leftmargin\labelsep
            \@openbib@code
            \usecounter{enumiv}%
            \let\p@enumiv\@empty
            \renewcommand\theenumiv{\@arabic\c@enumiv}}%
      \sloppy
      \clubpenalty4000
      \@clubpenalty \clubpenalty
      \widowpenalty4000%
      \sfcode`\.\@m}
     {\def\@noitemerr
       {\@latex@warning{Empty `thebibliography' environment}}%
      \endlist}

\renewcommand{\chaptermark}[1]{\markboth{#1}{}}
\lhead [\fancyplain {\bfseries\thepage} {\bfseries\thepage \\ \textcolor{myGray}{\rule{\textwidth}{2mm}}}]
       {\fancyplain {} {\bfseries\leftmark \\ \textcolor{myGray}{\rule{\textwidth}{2mm}}}}
\rhead [\fancyplain {} {\bfseries\leftmark \\ \textcolor{myGray}{\rule{\textwidth}{2mm}}}]
       {\fancyplain {\bfseries\thepage} {\bfseries\thepage \\ \textcolor{myGray}{\rule{\textwidth}{2mm}}}}
\chead{}
\cfoot{}
\lfoot [{\textcolor{myGray}{\rule{\textwidth}{2mm}}\\ \mbox{\scriptsize{\MyTitle}}}]
       {\textcolor{myGray}{\rule{\textwidth}{2mm}}\\ \mbox{\scriptsize{Olivier PANTALE}}}
\rfoot [{\textcolor{myGray}{\rule{\textwidth}{2mm}}\\ \mbox{\scriptsize{Olivier PANTALE}}}]
       {\textcolor{myGray}{\rule{\textwidth}{2mm}}\\ \mbox{\scriptsize{\MyTitle}}}
\renewcommand{\headrulewidth}{0pt}
\renewcommand{\footrulewidth}{0pt}

% cette ligne est necessaire si il n'y a pas d'algorithme dans le fichier lyx
\@ifundefined{algorithm}{

\newfloat{algorithm}{htbp}{loa}
}

% dfinitions pour les sections en couleur dans le texte
\renewcommand\section{\@startsection {section}{1}{\z@}%
                                   {-3.5ex \@plus -1ex \@minus -.2ex}%
                                   {2.3ex \@plus.2ex}%
                                   {\normalfont\LARGE\bfseries\color{myRed}}}
\renewcommand\subsection{\@startsection{subsection}{2}{\z@}%
                                     {-3.25ex\@plus -1ex \@minus -.2ex}%
                                     {1.5ex \@plus .2ex}%
                                     {\normalfont\Large\bfseries\color{myRed}}}
\renewcommand\subsubsection{\@startsection{subsubsection}{3}{\z@}%
                                     {-3.25ex\@plus -1ex \@minus -.2ex}%
                                     {1.5ex \@plus .2ex}%
                                     {\normalfont\large\bfseries\color{myRed}}}

% le super format pour les chapitres
\colorlet{chpnumbercolor}{black}
%\makeatletter
\let\oldl@chapter\l@chapter
\def\l@chapter#1#2{\oldl@chapter{#1}{\textcolor{chpnumbercolor}{#2}}}

\let\old@dottedcontentsline\@dottedtocline
\def\@dottedtocline#1#2#3#4#5{%
\old@dottedcontentsline{#1}{#2}{#3}{#4}{{\textcolor{chpnumbercolor}{#5}}}}
%\makeatother

\titleformat{\chapter}[display]
  {\normalfont\color{myRed}}
  {\filleft\Huge\sffamily\bfseries\chaptertitlename\hspace*{2mm}%
  \begin{tikzpicture}[baseline={([yshift=-.6ex]current bounding box.center)}]
    \node[fill=myRed, circle, text=white] {\thechapter};
  \end{tikzpicture}}
  {1ex}
  {\titlerule[5pt]\vspace*{2ex}\huge\sffamily\itshape}
  []

\titleformat{name=\chapter, numberless}[display]
  {\normalfont\color{myRed}}
  {}
  {1ex}
  {\vspace*{2ex}\huge\sffamily\itshape}
  []

% Command to print the actual minitoc
\newcommand{\printmyminitoc}[1][2]{%
    \vspace*{-5ex}
%    \noindent\hspace*{-0.5\hoffset}\hspace*{-0.5in}\hspace*{-0.5\oddsidemargin}%
    \colorlet{chpnumbercolor}{myRed}%
    \begin{tikzpicture}
    \node[rounded corners, align=left, fill=myGray80, inner sep=5mm]{%
        \color{myRed}%
        \begin{minipage}{0.95\columnwidth}%minipage trick
        \printcontents[chapters]{}{1}{\setcounter{tocdepth}{#1}}
        \end{minipage}};
    \end{tikzpicture}
    \vspace*{5ex}}

\def\@part[#1]#2{%
    \ifnum \c@secnumdepth >-2\relax
      \refstepcounter{part}%
      \addcontentsline{toc}{part}{\thepart\hspace{1em}#1}%
    \else
      \addcontentsline{toc}{part}{#1}%
    \fi
    \markboth{}{}%
    {\centering
     \interlinepenalty \@M
    \textcolor{myRed}{\rule{0.5\textwidth}{4mm}}%  <--- the rule
    \nobreak
    \vskip 5\p@
    \normalfont
     \ifnum \c@secnumdepth >-2\relax
       \huge\bfseries \color{myRed}\partname~\thepart
       \par
       \vskip 20\p@
     \fi
     \Huge \bfseries #2\par}%
    \vskip 5\p@
    \textcolor{myRed}{\rule{\textwidth}{4mm}}%  <--- the rule
    \nobreak
    \vskip 10\p@
    \@endpart}

\def\@endpart{\vfil\newpage
	\setcounter{chapter}{0} %reset the counter between chapters
              \if@twoside
               \if@openright
                \null
                \thispagestyle{empty}%
                \newpage
               \fi
              \fi
              \if@tempswa
                \twocolumn
              \fi}

% creation des programmes
\floatstyle{ruled}
\newfloat{Program}{thp}{lop}[chapter]

% plus de place dans les listes
\renewcommand*\l@figure{\@dottedtocline{1}{1.5em}{3.3em}}
\renewcommand*\l@table{\@dottedtocline{1}{1.5em}{3.3em}}
\renewcommand\listof[2]{%
  \@ifundefined{ext@#1}{\float@error{#1}}{%
    \@ifundefined{chapter}{\def\@tempa{\section*}}%
      {\def\@tempa{\chapter*}}%
    \@tempa{#2\@mkboth{\uppercase{#2}}{\uppercase{#2}}}%
    \@namedef{l@#1}{\@dottedtocline{1}{1.5em}{3.3em}}%
    \@starttoc{\@nameuse{ext@#1}}}}

\AtBeginDocument{
\addto\captionsfrench{
\renewcommand{\figurename}{Figure }
\renewcommand{\tablename}{Tableau }
}}

% les numros incluent le numro de la partie
\renewcommand \thefigure
     {\ifnum \c@part>\z@ \thepart.\fi \ifnum \c@chapter>\z@ \thechapter.\fi \@arabic\c@figure}
\renewcommand \thetable
     {\ifnum \c@part>\z@ \thepart.\fi \ifnum \c@chapter>\z@ \thechapter.\fi \@arabic\c@table}

%Definition des mises en page
\newcommand{\listingfontsscale}{\footnotesize}
\usepackage{esvect}
\usepackage{keystroke}
\usepackage{xstring}
\usepackage{transparent}
\usepackage{xspace}
\usepackage{color}
\usepackage{pdfpages}

% Define the set of colors
\definecolor{myGray}{gray}{0.75}
\definecolor{myGray90}{gray}{0.90}
\definecolor{myGray80}{gray}{0.80}
\definecolor{myGray50}{gray}{0.50}
\definecolor{myGray40}{gray}{0.40}
\definecolor{myRed}{rgb}{0.34,0.05,0.05}  
\definecolor{myLightRed}{rgb}{0.54,0.05,0.05}  
\definecolor{myDarkRed}{rgb}{0.14,0.05,0.05} 
\definecolor{myBlue}{rgb}{0.05,0.05,0.34}  % Blue color
\definecolor{myGreen}{rgb}{0.05,0.34,0.05}  % Green color

% Command to define two texts depending on the current language
\newcommand{\FrEn}[2]{\IfStrEq*{\languagename}{french}{#1}{#2}}

% Redefine the esvector shape to remove overlapping
\def\traitfill@#1#2#3#4{%
  $\m@th\mkern2mu\relax#4#1\mkern-1.5mu %on met \relbaredd au d\'ebut
   \cleaders\hbox{$#4\mkern-0.3mu#2\mkern-0.3mu$}\hfill %remplit avec relbareda
   \mkern-1.5mu#3$%
}

% affect overrightarrow to vv
\renewcommand{\overrightarrow}{\vv}

% définition des commandes locales
% Definition of the C++ special command
\DeclareRobustCommand{\Cpp}
{\valign{\vfil\hbox{##}\vfil\cr 
   \textsf{C\kern-.1em}\cr
   $\hbox{\fontsize{\ssf@size}{0}\textbf{+\kern-0.05em+}}$\cr}%
}

% Definition des listings avec langage par défaut Python
\usepackage{listings}

\lstnewenvironment{CppListing}[1][]
    {\lstset{language=C++,
	keywordstyle=\color{myRed},
	numberstyle=\footnotesize\color{myGray50},
	numbers=left,
	frame=single,xleftmargin=5.0ex,xrightmargin=1.0ex,
	literate={-} { \textrm{-}}{1},
	%otherkeywords={>>>},
	%morekeywords={as,assert,with,yield},
	commentstyle=\color{myGray40},
	stringstyle=\color{myGreen},
	basicstyle=\listingfontsscale\sffamily\upshape,
	escapeinside={(*}{*)},  
	backgroundcolor=\color{myGray90},
	rulecolor=\color{myGray50},
        #1}%
        \noindent\begin{minipage}{\textwidth}\csname\@lst @SetFirstNumber\endcsname\end{minipage}}
        {\csname\@lst @SaveFirstNumber\endcsname}

\lstnewenvironment{FortranListing}[1][]
    {\lstset{language=[77]Fortran,
	keywordstyle=\color{myRed},
	numberstyle=\footnotesize\color{myGray50},
	numbers=left,
	frame=single,xleftmargin=5.0ex,xrightmargin=1.0ex,
	literate={-} { \textrm{-}}{1},
	%otherkeywords={>>>},
	%morekeywords={as,assert,with,yield},
	commentstyle=\color{myGray50},
	stringstyle=\color{myGreen},
	basicstyle=\listingfontsscale\sffamily\upshape,
	escapeinside={(*}{*)},  
	backgroundcolor=\color{myGray90},
	rulecolor=\color{myGray50},
        #1}%
        \noindent\begin{minipage}{\textwidth}\csname\@lst @SetFirstNumber\endcsname\end{minipage}}
        {\csname\@lst @SaveFirstNumber\endcsname}

\lstnewenvironment{PythonListing}[1][]
    {\lstset{language=Python,
	keywordstyle=\color{myRed},
	numberstyle=\footnotesize\color{myGray50},
	numbers=left,
	frame=single,xleftmargin=5.0ex,xrightmargin=1.0ex,
	%float=h,
	literate={-} { \textrm{-}}{1},
	%otherkeywords={>>>},
	morekeywords={as,assert,with,yield},
	commentstyle=\color{myGray40},
	stringstyle=\color{myGreen},
	basicstyle=\listingfontsscale\sffamily\upshape,
	escapeinside={(*}{*)},  
	backgroundcolor=\color{myGray90},
	rulecolor=\color{myGray50},
        #1}%
        \noindent\begin{minipage}{\textwidth}\csname\@lst @SetFirstNumber\endcsname\end{minipage}}
        {\csname\@lst @SaveFirstNumber\endcsname}

\lstnewenvironment{AbaqusListing}[1][]
    {\lstset{language=Python,
	keywordstyle=\color{myRed},
	numberstyle=\footnotesize\color{myGray50},
	numbers=left,
	frame=single,xleftmargin=5.0ex,xrightmargin=1.0ex,
	literate={-} { \textrm{-}}{1},
	%otherkeywords={>>>},
	morekeywords={as,assert,with,yield},
	commentstyle=\color{myGray40},
	stringstyle=\color{myGreen},
	basicstyle=\listingfontsscale\sffamily\upshape,
	escapeinside={(*}{*)},  
	backgroundcolor=\color{myGray90},
	rulecolor=\color{myGray50},
		#1}%
        \noindent\begin{minipage}{\textwidth}\csname\@lst @SetFirstNumber\endcsname\end{minipage}}
        {\csname\@lst @SaveFirstNumber\endcsname}

\lstnewenvironment{BashListing}[1][]
	{\lstset{language=bash,
		keywordstyle=\color{myRed},
		numberstyle=\footnotesize\color{myGray50},
		numbers=left,
		frame=single,xleftmargin=5.0ex,xrightmargin=1.0ex,
		literate={-} { \textrm{-}}{1},
		%otherkeywords={>>>},
		%morekeywords={as,assert,with,yield},
		commentstyle=\color{myGray40},
		stringstyle=\color{myGreen},
		basicstyle=\sffamily\listingfontsscale\upshape,
		escapeinside={(*}{*)},  
		backgroundcolor=\color{myGray90},
		rulecolor=\color{myGray50},
		#1}%
	\noindent\begin{minipage}{\textwidth}\csname\@lst @SetFirstNumber\endcsname\end{minipage}}
{\csname\@lst @SaveFirstNumber\endcsname}

\lstset{language=Python,
keywordstyle=\color{myRed},
numberstyle=\footnotesize\color{myGray40},
numbers=left,
frame=single,xleftmargin=5.0ex,xrightmargin=1.0ex,
literate={-} { \textrm{-}}{1},
%otherkeywords={>>>},
morekeywords={as,assert,with,yield},
commentstyle=\color{myGray50},
stringstyle=\color{myGreen},
basicstyle=\listingfontsscale\sffamily\upshape,
escapeinside={(*}{*)},  
backgroundcolor=\color{white},
rulecolor=\color{myGray50}}
\newcommand{\pyout}[1]{\color{blue}\textsl{#1}}

\newcommand{\TI}[1]{\mbox{\large\ensuremath{\mathsf{#1}}}}
\newcommand{\TII}[1]{\mbox{\large\ensuremath{\mathbb{#1}}}}
\newcommand{\TiI}[1]{\mbox{\large\ensuremath{\mathcal{#1}}}}

%Domaines d'intégration
\newcommand{\Ve}	{\Omega_{x}^{e}}
\newcommand{\Se}	{\Gamma_{x}^{e}}

%Tenseur des contraintes, déformations, etc...
\newcommand{\Sig}	{\mbox{\large\ensuremath{\boldsymbol{\sigma}}}}
\newcommand{\Eps}	{\mbox{\large\ensuremath{\boldsymbol{\varepsilon}}}}
\newcommand{\Alp}	{\mbox{\large\ensuremath{\boldsymbol{\alpha}}}}
\newcommand{\ro}	{\mbox{\large\ensuremath{\boldsymbol{\rho}}}}
\newcommand{\gam}	{\mbox{\large\ensuremath{\boldsymbol{\gamma}}}}
\newcommand{\Gam}	{\mbox{\large\ensuremath{\boldsymbol{\Gamma}}}}
\newcommand{\om}	{\mbox{\large\ensuremath{\boldsymbol{\omega}}}}
\newcommand{\Om}	{\mbox{\large\ensuremath{\boldsymbol{\Omega}}}}
\newcommand{\Fi}	{\mbox{\Large\ensuremath{\boldsymbol{\phi}}}}
\newcommand{\Dev}	{\mbox{\Large\ensuremath{\mathsf{s}}}}

%Tenseur Identité
\newcommand{\IId}	{\mbox{\large\ensuremath{\mathbb{I}}}}
\newcommand{\Iid}	{\mbox{\Large\ensuremath{\mathcal{I}}}}
\newcommand{\Id}  	{\mbox{\large\ensuremath{\mathsf{1}}}}

% Tenseurs A du second, 3eme et 4eme ordre
\newcommand{\A}  	{\mbox{\large\ensuremath{\mathsf{A}}}}
\newcommand{\IiA}	{\mbox{\large\ensuremath{\mathcal{A}}}}
\newcommand{\IIA}	{\mbox{\large\ensuremath{\mathbb{A}}}}
\newcommand{\ia}	{\mbox{\large\ensuremath{\mathsf{a}}}}

%Tenseurs B du second, 3eme et 4eme ordre
\newcommand{\B}		{\mbox{\large\ensuremath{\mathsf{B}}}}
\newcommand{\IiB}	{\mbox{\large\ensuremath{\mathcal{B}}}}
\newcommand{\IIB}	{\mbox{\large\ensuremath{\mathbb{B}}}}
\newcommand{\ib}	{\mbox{\large\ensuremath{\mathsf{b}}}}

% Tenseurs C du second, 3eme et 4eme ordre
\newcommand{\C}		{\mbox{\large\ensuremath{\mathsf{C}}}}
\newcommand{\IiC}	{\mbox{\large\ensuremath{\mathcal{C}}}}
\newcommand{\IIC}	{\mbox{\large\ensuremath{\mathbb{C}}}}

% Tenseurs D du second, 3eme et 4eme ordre
\newcommand{\D}		{\mbox{\large\ensuremath{\mathsf{D}}}}
\newcommand{\IiD}	{\mbox{\large\ensuremath{\mathcal{D}}}}
\newcommand{\IID}	{\mbox{\large\ensuremath{\mathbb{D}}}}

% Tenseurs E du second, 3eme et 4eme ordre
\newcommand{\E}		{\mbox{\large\ensuremath{\mathsf{E}}}}
\newcommand{\IiE}	{\mbox{\large\ensuremath{\mathcal{E}}}}
\newcommand{\IIE}	{\mbox{\large\ensuremath{\mathbb{E}}}}
\newcommand{\e}		{\mbox{\large\ensuremath{\mathsf{e}}}}

% Tenseurs F du second, 3eme et 4eme ordre
\newcommand{\F}		{\mbox{\large\ensuremath{\mathsf{F}}}}
\newcommand{\IiF}	{\mbox{\large\ensuremath{\mathcal{F}}}}
\newcommand{\IIF}	{\mbox{\large\ensuremath{\mathbb{F}}}}
\newcommand{\f}		{\mbox{\large\ensuremath{\mathsf{f}}}}

% Tenseurs H du second, 3eme et 4eme ordre
\newcommand{\G}		{\mbox{\large\ensuremath{\mathsf{G}}}}
\newcommand{\IiG}	{\mbox{\large\ensuremath{\mathcal{G}}}}
\newcommand{\IIG}	{\mbox{\large\ensuremath{\mathbb{G}}}}
\newcommand{\ig}	{\mbox{\large\ensuremath{\mathsf{g}}}}

% Tenseurs H du second, 3eme et 4eme ordre
\newcommand{\iH}	{\mbox{\large\ensuremath{\mathsf{H}}}}
\newcommand{\IiH}	{\mbox{\large\ensuremath{\mathcal{H}}}}
\newcommand{\IIH}	{\mbox{\large\ensuremath{\mathbb{H}}}}

% Tenseurs J du second, 3eme et 4eme ordre
\newcommand{\J}		{\mbox{\large\ensuremath{\mathsf{J}}}}
\newcommand{\IiJ}	{\mbox{\large\ensuremath{\mathcal{J}}}}
\newcommand{\IIJ}	{\mbox{\large\ensuremath{\mathbb{J}}}}

% Tenseurs K du second, 3eme et 4eme ordre
\newcommand{\K}		{\mbox{\large\ensuremath{\mathsf{K}}}}
\newcommand{\IiK}	{\mbox{\large\ensuremath{\mathcal{K}}}}
\newcommand{\IIK}	{\mbox{\large\ensuremath{\mathbb{K}}}}

% Tenseurs L du second, 3eme et 4eme ordre
\newcommand{\iL}	{\mbox{\large\ensuremath{\mathsf{L}}}}
\newcommand{\IiL}	{\mbox{\large\ensuremath{\mathcal{L}}}}
\newcommand{\IIL}	{\mbox{\large\ensuremath{\mathbb{L}}}}

% Tenseurs M du second, 3eme et 4eme ordre
\newcommand{\M}		{\mbox{\large\ensuremath{\mathsf{M}}}}
\newcommand{\IiM}	{\mbox{\large\ensuremath{\mathcal{M}}}}
\newcommand{\IIM}	{\mbox{\large\ensuremath{\mathbb{M}}}}
\newcommand{\m}		{\mbox{\large\ensuremath{\mathsf{m}}}}

% Tenseurs N du second, 3eme et 4eme ordre
\newcommand{\N}		{\mbox{\large\ensuremath{\mathsf{N}}}}
\newcommand{\IiN}	{\mbox{\large\ensuremath{\mathcal{N}}}}
\newcommand{\IIN}	{\mbox{\large\ensuremath{\mathbb{N}}}}
\newcommand{\n}		{\mbox{\large\ensuremath{\mathsf{n}}}}

% Tenseurs P du second, 3eme et 4eme ordre
\newcommand{\iP}	{\mbox{\large\ensuremath{\mathsf{P}}}}
\newcommand{\IiP}	{\mbox{\large\ensuremath{\mathcal{P}}}}
\newcommand{\IIP}	{\mbox{\large\ensuremath{\mathbb{P}}}}
\newcommand{\p}		{\mbox{\large\ensuremath{\mathsf{p}}}}

% Tenseurs Q du second, 3eme et 4eme ordre
\newcommand{\Q}		{\mbox{\large\ensuremath{\mathsf{Q}}}}
\newcommand{\IiQ}	{\mbox{\large\ensuremath{\mathcal{Q}}}}
\newcommand{\IIQ}	{\mbox{\large\ensuremath{\mathbb{Q}}}}
\newcommand{\q}		{\mbox{\large\ensuremath{\mathsf{q}}}}

% Tenseurs R du second, 3eme et 4eme ordre
\newcommand{\R}		{\mbox{\large\ensuremath{\mathsf{R}}}}
\newcommand{\IiR}	{\mbox{\large\ensuremath{\mathcal{R}}}}
\newcommand{\IIR}	{\mbox{\large\ensuremath{\mathbb{R}}}}
\newcommand{\ir}	{\mbox{\large\ensuremath{\mathsf{r}}}}

% Tenseurs S du second, 3eme et 4eme ordre
\newcommand{\iS}	{\mbox{\large\ensuremath{\mathsf{S}}}}
\newcommand{\IIS}	{\mbox{\large\ensuremath{\mathbb{S}}}}
\newcommand{\IiS}	{\mbox{\Large\ensuremath{\mathcal{S}}}}

% Tenseurs T du second, 3eme et 4eme ordre
\newcommand{\T}		{\mbox{\large\ensuremath{\mathsf{T}}}}
\newcommand{\IiT}	{\mbox{\large\ensuremath{\mathcal{T}}}}
\newcommand{\IIT}	{\mbox{\large\ensuremath{\mathbb{T}}}}
\newcommand{\bt}	{\mbox{\large\ensuremath{\mathsf{t}}}}

% Tenseurs U du second, 3eme et 4eme ordre
\newcommand{\U}		{\mbox{\large\ensuremath{\mathsf{U}}}}
\newcommand{\IiU}	{\mbox{\large\ensuremath{\mathcal{U}}}}
\newcommand{\IIU}	{\mbox{\large\ensuremath{\mathbb{U}}}}
\newcommand{\iu}	{\mbox{\large\ensuremath{\mathsf{u}}}}

% Tenseurs V du second, 3eme et 4eme ordre
\newcommand{\V}		{\mbox{\large\ensuremath{\mathsf{V}}}}
\newcommand{\IiV}	{\mbox{\large\ensuremath{\mathcal{V}}}}
\newcommand{\IIV}	{\mbox{\large\ensuremath{\mathbb{V}}}}
\newcommand{\iv}	{\mbox{\large\ensuremath{\mathsf{v}}}}

% Tenseurs W du second, 3eme et 4eme ordre
\newcommand{\W}		{\mbox{\large\ensuremath{\mathsf{W}}}}
\newcommand{\IiW}	{\mbox{\large\ensuremath{\mathcal{W}}}}
\newcommand{\IIW}	{\mbox{\large\ensuremath{\mathbb{W}}}}
\newcommand{\iw}	{\mbox{\large\ensuremath{\mathsf{w}}}}

% Tenseurs X du second, 3eme et 4eme ordre
\newcommand{\X}		{\mbox{\large\ensuremath{\mathsf{X}}}}
\newcommand{\IiX}	{\mbox{\large\ensuremath{\mathcal{X}}}}
\newcommand{\x}		{\mbox{\large\ensuremath{\mathsf{x}}}}
\newcommand{\IIX}	{\mbox{\large\ensuremath{\mathbb{X}}}}

% Tenseurs Y du second, 3eme et 4eme ordre
\newcommand{\Y}		{\mbox{\large\ensuremath{\mathsf{Y}}}}
\newcommand{\IiY}	{\mbox{\large\ensuremath{\mathcal{Y}}}}
\newcommand{\y}		{\mbox{\large\ensuremath{\mathsf{y}}}}
\newcommand{\IIY}	{\mbox{\large\ensuremath{\mathbb{Y}}}}

% Tenseurs Z du second, 3eme et 4eme ordre
\newcommand{\Z}		{\mbox{\large\ensuremath{\mathsf{Z}}}}
\newcommand{\IiZ}	{\mbox{\large\ensuremath{\mathcal{Z}}}}
\newcommand{\z}		{\mbox{\large\ensuremath{\mathsf{z}}}}
\newcommand{\IIZ}	{\mbox{\large\ensuremath{\mathbb{Z}}}}

%Null tensor
\newcommand{\0}		{\mbox{\large\ensuremath{\mathsf{0}}}}

%Commandes spéciales espacements
\newcommand{\Vs}	{\vspace{1cm}}
\newcommand{\vs}	{\vspace{2mm}}

%Commandes spéciales
\newcommand{\tr}	{\mbox{\ensuremath{\mathsf{\,tr\,}}}}
\newcommand{\dev}	{\mbox{\ensuremath{\mathsf{\,dev\,}}}}
\newcommand{\grad}	{\mbox{\ensuremath{\mathsf{\,grad\,}}}}
\newcommand{\Div}	{\mbox{\ensuremath{\mathsf{\,div\,}}}}
\newcommand{\curl}	{\mbox{\ensuremath{\mathsf{\,curl\,}}}}
\renewcommand{\det}	{\mbox{\ensuremath{\mathsf{\,det\,}}}}

\newcommand{\pluseq}{\mathrel{+}=}
\newcommand{\minuseq}{\mathrel{-}=}
\newcommand{\multeq}{\mathrel{*}=}
\newcommand{\diveq}{\mathrel{/}=}

\DeclareMathAlphabet{\mathbxsl}{\encodingdefault}{\rmdefault}{m}{it}
\newcommand{\invI}[1]	{\mbox{\large\ensuremath{\mathbxsl{I_{\mathsf{#1}}}}}}
\newcommand{\invII}[1]	{\mbox{\large\ensuremath{\mathbxsl{II_{\mathsf{#1}}}}}}
\newcommand{\invIII}[1]	{\mbox{\large\ensuremath{\mathbxsl{III_{\mathsf{#1}}}}}}

\newcommand*{\ie}	{\emph{i.e.}\@\xspace}
\newcommand*{\versus}	{\emph{vs.}\@\xspace}
\newcommand*{\eal}	{et \emph{al.}\@\xspace}
\newcommand*{\eg}	{e.g.,\@\xspace}

% Declaration des colorboxes
\RequirePackage{calc}
\newsavebox{\boitetitre}
\newlength{\tempdim}
\newlength{\largeurboitetitre}
\newlength{\hauteurboitetitre}
\newlength{\largeurtitre}

\newcommand{\espacetitre}{1.5}
\newcommand{\decalagetitreg}{1}
\newcommand{\decalagetitred}{5}

\newcommand{\traitressort}[2][1]{%
  \leaders\hrule height#2\hskip0pt plus #1fill\relax}

\newcommand{\traitpasressort}[2][1]{%
  \leaders\hrule height#2\hskip1.5em\relax}

\newcommand{\fixedtitlebox}[3][-0.3ex]{%
  \begin{lrbox}{\boitetitre}\kern\fboxsep\colorbox{myGray80}{#3}\kern\fboxsep\end{lrbox}%
  \settowidth{\largeurboitetitre}{\usebox{\boitetitre}}%
  \settowidth{\largeurtitre}{#2}%
  \settoheight{\hauteurboitetitre}{\usebox{\boitetitre}}%
  \settodepth{\tempdim}{\usebox{\boitetitre}}%
  \addtolength{\hauteurboitetitre}{\tempdim+2\fboxrule+2\fboxrule+2\fboxsep}%
  \parbox{2\fboxrule}{%
    \rule{2\fboxrule}{\hauteurboitetitre}}%
  \parbox{\largeurboitetitre}{%
    \begin{flushleft}
      \makebox[\largeurboitetitre]{%
        \traitpasressort[\decalagetitreg]{2\fboxrule}%
        \raisebox{#1}[0pt][0pt]{%
          \kern\espacetitre\fboxsep\textbf{#2}\kern\espacetitre\fboxsep}%
        \traitressort[\decalagetitred]{2\fboxrule}}\\[\fboxsep]\nointerlineskip
      \usebox{\boitetitre}\\[\fboxsep]\nointerlineskip%
      \rule{\largeurboitetitre}{2\fboxrule}
    \end{flushleft}}%
  \parbox{\fboxrule}{%
    \rule{2\fboxrule}{\hauteurboitetitre}}}

\newcommand{\titlebox}[3][-0.3ex]{%
  \begin{lrbox}{\boitetitre}\kern\fboxsep\colorbox{myGray80}{#3}\kern\fboxsep\end{lrbox}%
  \settowidth{\largeurboitetitre}{\usebox{\boitetitre}}%
  \settowidth{\largeurtitre}{#2}%
  \settoheight{\hauteurboitetitre}{\usebox{\boitetitre}}%
  \settodepth{\tempdim}{\usebox{\boitetitre}}%
  \addtolength{\hauteurboitetitre}{\tempdim+2\fboxrule+2\fboxrule+2\fboxsep}%
  \parbox{2\fboxrule}{%
    \rule{2\fboxrule}{\hauteurboitetitre}}%
  \parbox{\largeurboitetitre}{%
    \begin{flushleft}
      \makebox[\largeurboitetitre]{%
        \traitressort[\decalagetitreg]{2\fboxrule}%
        \raisebox{#1}[0pt][0pt]{%
          \kern\espacetitre\fboxsep\textbf{#2}\kern\espacetitre\fboxsep}%
        \traitressort[\decalagetitred]{2\fboxrule}}\\[\fboxsep]\nointerlineskip
      \usebox{\boitetitre}\\[\fboxsep]\nointerlineskip%
      \rule{\largeurboitetitre}{2\fboxrule}
    \end{flushleft}}%
  \parbox{\fboxrule}{%
    \rule{2\fboxrule}{\hauteurboitetitre}}}



% Symboles des drives
\renewcommand{\bullet}{\begin{picture}(3,2)\put(1.5,1){\circle*{2}}\end{picture}}

\ifx\pdfoutput\undefined
\usepackage[ps2pdf,pagebackref=true,colorlinks=true,linkcolor=myRed,citecolor=myRed]{hyperref}
\else
\usepackage[pdftex,pagebackref=true,colorlinks=true,linkcolor=myRed,citecolor=myRed]{hyperref}
\fi
\makeindex

\newcommand{\DynELA}{DynELA Finite Element Code}

\excludecomment{fullVersion}

\makeatother
